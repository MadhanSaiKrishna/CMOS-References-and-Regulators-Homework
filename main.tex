\documentclass[a4paper]{article}

%% Language and font encodings
\usepackage[english]{babel}
\usepackage[utf8x]{inputenc}
\usepackage[T1]{fontenc}

%% Page size and margins
\usepackage[a4paper,top=3cm,bottom=2cm,left=3cm,right=3cm,marginparwidth=1.75cm]{geometry}

%% Packages
\usepackage{amsmath}
\usepackage{bm}
\usepackage{graphicx}
\usepackage{algorithm}
\usepackage{float}
\usepackage{caption}
\usepackage{subcaption}
\usepackage{enumerate}
\usepackage[noend]
{algpseudocode}
\setlength{\marginparwidth}{2cm}
\usepackage[colorinlistoftodos]{todonotes}
\usepackage[colorlinks=false, allcolors=blue]{hyperref}
\usepackage{textcomp}
\usepackage{xcolor}


\title{\textbf{CMOS References and Regulators: Homework}}
\author{Chamarthy Madhan Sai Krishna\\2023102030}

\begin{document}

\maketitle

\tableofcontents

\section{Lecture 1}

\subsection{\colorbox{yellow}{Proving Resistance: Current is proportional to Voltage}}
Do the 'DC Analysis' for the following circuits. Plot V vs I $\&$ write expression for the slope. Prove that its a resistance.\\
\begin{figure}[H]
    \centering
    \includegraphics[width=0.8\linewidth]{images/Lec_1_Q_1.jpeg}
    \caption{Lecture 1 - Question 1 - (a) and (b)}
\end{figure}
\textit{Answer: }\\
Left graph shows the I - V characteristics of the device. Clearly, $I \propto V$. The right graph shows the expression for the slope. \\
Plot for (a) : Using a Voltage source \\
Plot for (b) : Using a Current source

\begin{figure}[H]
    \centering
    \includegraphics[width=1\linewidth]{images/Lec_1_Q_1_testbench.png}
    \caption{Lec 1: Q1: Testbench for (a) and (b)}
\end{figure}

\begin{figure}[H]
    \centering
    \includegraphics[width=1\linewidth]{images/Lec_1_Q_1_a_plot.png}
    \caption{Lec 1: Q1: Plot for (a)}
\end{figure}

\begin{figure}[H]
    \centering
    \includegraphics[width=1\linewidth]{images/Lec_1_Q_1_b_plot.png}
    \caption{Lec 1: Q1: Plot for (b)}
\end{figure}

\subsection{\colorbox{yellow}{Capacitance Amplifier using Controlled Sources}}
Design a capacitance amplifier using controlled sources.\\
\begin{figure}[H]
    \centering
    \includegraphics[width=0.8\linewidth]{images/Lec_1_Q_2_3_4.jpeg}
    \caption{Lecture 1 - Question 2, 3, 4}
\end{figure}
\textit{Answer: }\\
\begin{figure}[H]
    \centering
    \includegraphics[width=1\linewidth]{images/Lec_1_Q2_Soln.jpeg}
    \caption{Lecture 1 - Question 2 - Solution}
\end{figure}
\subsection{\colorbox{yellow}{Inductance Amplifier using Controlled Sources}}
Design a inductance amplifier using controlled sources.\\

\textit{Answer: }\\

\begin{figure}[H]
    \centering
    \includegraphics[width=1\linewidth]{images/Lec_1_Q3_Soln.jpeg}
    \caption{Lecture 1 - Question 3 - Solution}
\end{figure}

\subsection{\colorbox{yellow}{Resistance Amplifier using Controlled Sources}}
Sweep V vs I for (i) $k>1$ (ii) $0<k<1$ (iii) $K<0$ Prove $R_{eff}$ in each case is $\frac{R}{1-k}$.\\
\textit{Answer: }\\
\begin{figure}[H]
    \centering
    \includegraphics[width=1\linewidth]{images/Lec_1_Q4_Soln.jpeg}
    \caption{Lecture 1 - Question 4 - Solution}
\end{figure}

\section{Lecture 2}
\subsection{\colorbox{yellow}{Output Resistance}}
find the effective output resistance at $V_{out}$ i.e., $R_{out} = ?$ \\
\begin{figure}[H]
    \centering
    \includegraphics[width=1\linewidth]{images/Lec_2_Q_1_Soln.jpeg}
    \caption{Lec 2: Question 1}
\end{figure}

\begin{figure}[H]
    \centering
    \includegraphics[width=0.8\linewidth]{images/Lec_2_Q1.jpeg}
    \caption{Lecture 2 - Question 1 - Solution}
\end{figure}
\textit{Answer:}\\
When we nullify the V1, the vccs vanish and the effective output resistance at vout node is $R_{out} = r_{ds_1} || r_{ds_2}$. 

Proved this using the plot with values.

\begin{figure}[H]
    \centering
    \includegraphics[width=1\linewidth]{images/Lec_2_Q_1_testbench.png}
    \caption{Lec 2: Q1: Testbench}
\end{figure}

\subsection{\colorbox{yellow}{Step Response for RC Circuit}}
Plot $V_{out}$ vs time in Cadence. Prove Theory. \\
\begin{figure}[H]
    \centering
    \includegraphics[width=0.8\linewidth]{images/Lec_2_Q_2.jpeg}
    \caption{Lecture 2 - Question 2}
\end{figure}
\textit{Answer:}\\
Copy this from the notes\\
\begin{figure}[H]
    \centering
    \includegraphics[width=1\linewidth]{images/Lec_2_Q2_Soln_Part_1.jpeg}
    \caption{Lec 2: Solution for Question 2}
\end{figure}

\begin{figure}[H]
    \centering
    \includegraphics[width=1\linewidth]{images/Lec_2_Q2_Soln_Part_2.jpeg}
    \caption{Lec 2: Solution for Question 2}
\end{figure}

\begin{figure}[H]
    \centering
    \includegraphics[width=1\linewidth]{images/Lec_2_Q2_Soln_Part_3.jpeg}
    \caption{Lec 2: Solution for Question 2}
\end{figure}
Proved this using theory and from the simulations  by finding the output voltage after 1 time constant.

\begin{figure}[H]
    \centering
    \includegraphics[width=1\linewidth]{images/Lec_2_Q_2_testbench.png}
    \caption{Lec 2: uestion 2 Testbench}
\end{figure}

\begin{figure}[H]
    \centering
    \includegraphics[width=1\linewidth]{images/Lec_2_Q_2_a.png}
    \caption{Lec 2: Q2: Soln - Part (a)}
\end{figure}

\begin{figure}[H]
    \centering
    \includegraphics[width=1\linewidth]{images/Lec_2_Q_2_b.png}
    \caption{Lec 2 : Q2: Soln - Part (b)}
\end{figure}

\section{Lecture 3}
\subsection{\colorbox{yellow}{Step Response for passive RLC circuit Combinations}}
Plot $V_{out}$ vs $time$ and $I_{out}$ vs $time$
\begin{itemize}
    \item 
    \item 
    \item 
\end{itemize}
\begin{figure}[H]
    \centering
    \includegraphics[width=0.8\linewidth]{images/Lec_3_Q_1.jpeg}
    \caption{Lecture 3 - Question 1}
\end{figure}

\begin{figure}[H]
    \centering
    \includegraphics[width=1\linewidth]{images/Lec_3_Q1_Soln_1.jpeg}
    \caption{Lec 3: Solution for Question 1}
\end{figure}

\begin{figure}[H]
    \centering
    \includegraphics[width=1\linewidth]{images/Lec_3_Q1_Soln_2.jpeg}
    \caption{Lec 3: Solution for Question 1}
\end{figure}


\subsection{\colorbox{yellow}{Pulse Response for RC Circuit with varying Duty Cycle and Time Period}}
Plot $V_{out}$ vs $time$ (a) $D = 10\%$ (b) $D = 90\%$ (i) $T = 0.1RC$ (ii) $T = 10RC$ where $D = \frac{T_{ON}}{T}$
\begin{figure}[H]
    \centering
    \includegraphics[width=0.8\linewidth]{images/Lec_3_Q_2.jpeg}
    \caption{Lecture 3 - Question 2}
\end{figure}

\begin{figure}[H]
    \centering
    \includegraphics[width=1\linewidth]{images/Lec_3_Q2_Soln.jpeg}
    \caption{Lec 3: Solution for Question 2}
\end{figure}
\subsection{\colorbox{yellow}{Bode Plot for a given Transfer Function}}
Draw Bode plot for $H(s) = \frac{(s - z_1)(s+z_2)}{(s+P_1)(s+P_2)(s+P_3)}$ where (i) $z_2 << p_1 << p_2 << p_3 << z_1$ (ii) $p_3 << p_2 << p_1 << z_1 << z_2$
\begin{figure}[H]
    \centering
    \includegraphics[width=0.8\linewidth]{images/Lec_3_Q_3.jpeg}
    \caption{Lecture 3 - Question 3}
\end{figure}
\begin{figure}[H]
    \centering
    \includegraphics[width=1\linewidth]{images/Lec_3_Q3_Soln.jpeg}
    \caption{Lec 3: Solution for Question 3}
\end{figure}
\section{Lecture 4}
\subsection{\colorbox{yellow}{Non-inverting Schimtt Trigger}}
Design a non-inverting schimtt trigger using opamp based circuits.
\begin{figure}[H]
    \centering
    \includegraphics[width=0.8\linewidth]{images/Lec_4_Q_1.jpeg}
    \caption{Lecture 4 - Question 1}
\end{figure}
\begin{figure}[H]
    \centering
    \includegraphics[width=1\linewidth]{images/Lec_4_Q1_Soln.jpeg}
    \caption{Lec 4: Solution for Question 1}
\end{figure}
\subsection{\colorbox{green}{Expression for Closed loop Unity gain bandwidth}}
Find the expression for Closed loop Unity gain bandwidth.
\begin{figure}[H]
    \centering
    \includegraphics[width=0.8\linewidth]{images/Lec_4_Q_2_3.jpeg}
    \caption{Lecture 4 - Question 2,3}
\end{figure}
\textbf{Answer:}\\
Assuming a single pole system, the below image shows the closed loop unity gain bandwidth.\\
\begin{figure}[H]
    \centering
    \includegraphics[width=1\linewidth]{images/Lec_4_Q2_Soln.jpeg}
    \caption{Lec 4: Solution for Question 2}
\end{figure}
\subsection{\colorbox{yellow}{Non-Inverting Amplifier using VCVS}}
In cadence, Plot $V_o$ vs $V_{in}$ (Sweep $V_{in}$ from $-V_{ss}$ to $V_{DD}$) for non-inverting amplifier using Opamp and VCVS for both positive and -ve feedback.

\textbf{Answer:}\\
A non--inverting amplifier using postivie feedback is not possible. Due to positive feedback, the output will saturate to either $V_{DD}$ or $V_{SS}$ for any non-zero input.\\
For negative feedback, the output voltage is given by\\ $V_{out} = V_{in} \cdot (1 + \frac{R_2}{R_1})$.\\
For the non-inverting amplifier we can use either an opamp or a VCVS.\\
\begin{figure}[H]
    \centering
    \includegraphics[width=0.8\linewidth]{images/Lec_4_Q3_Soln.png}
    \caption{Lecture 4 - Question 3 Solution: Non-inverting Amplifier using VCVS}
\end{figure}
\section{Lecture 5}
\subsection{\colorbox{green}{Single Pole Opamp with Feedback - Bode Plot}}
In Cadence, Bode Plot for 
\begin{enumerate}
    \item A = 100, check $V_x$ (offset) [DC]
    \item A = 1000, check $V_x$ (offset) [DC]
    \item UGB = 1MHz, Gain = 1000, Design $G_m$, $R_{out}$, $R_{1}$, $R_{2}$, $C$, $V_{Ref} = 0.6V$, $V_{out} = 1.2V$
\end{enumerate}
\begin{figure}[H]
    \centering
    \includegraphics[width=0.8\linewidth]{images/Lec_5_Q_1.jpeg}
    \caption{Lecture 5 - Question 1}
\end{figure}
\textbf{Answer:}\\
\begin{figure}[H]
    \centering
    \includegraphics[width=1\linewidth]{images/Lec_5_Q_1_Soln_1.jpg}
    \caption{Lecture 5 - Question 1 - Theory}
\end{figure}

\begin{figure}[H]
    \centering
    \includegraphics[width=1\linewidth]{images/Lec_5_Q_1_Soln_2.jpg}
    \caption{Lecture 5 - Question 2 - Theory}
\end{figure}

\begin{figure}[H]
    \centering
    \includegraphics[width=1\linewidth]{images/Lec_5_Q1_1_Soln.png}
    \caption{Lec 5: Solution for Question 1\_1)}
\end{figure}
\begin{figure}[H]
    \centering
    \includegraphics[width=1\linewidth]{images/Lec_5_Q1_2_Soln.png}
    \caption{Lec 5: Solution for Question 1\_2}
\end{figure}

\subsection{\colorbox{gray}{Bode Plot for a 2 Pole System}}
In cadence, AC analysis. Bode Plot for $\frac{V_{out}}{V_{{in}}}$
\begin{enumerate}
    \item $G_{m_1} = G_{m_2} = 10 \mu F$ , $R_1 = R_2 = 10M \Omega$ , $C_1 = 10fF, C_2 = 2fF$
    \item add $C_{effective}$ at $C_1$ and get phase margin = $45^\circ$
\end{enumerate}
\begin{figure}[H]
    \centering
    \includegraphics[width=0.8\linewidth]{images/Lec_5_Q_2_ckt.jpeg}
    \caption{Lecture 5 - Question 2 - circuit}
\end{figure}
\begin{figure}[H]
    \centering
    \includegraphics[width=1\linewidth]{images/Lec_5_Q_2_ques.jpeg}
    \caption{Lecture 5 - Question 2 - Question}
\end{figure}
\textbf{Answer:}\\
\textbf{Given:}
\[
g_m = 10~\mu\text{S}, \quad R_1 = R_2 = 10~\text{M}\Omega, \quad 
C_1 = 10~\text{fF}, \quad C_2 = 2~\text{fF}
\]

\textbf{(1) DC Gain and Pole Locations:}
\[
A_1 = A_2 = g_m R = 10\times10^{-6}\times10\times10^{6} = 100
\]
\[
A_0 = A_1A_2 = 100\times100 = 10^4 = 80~\text{dB}
\]
\[
\omega_{p1} = \frac{1}{R_1C_1} = 1\times10^7~\text{rad/s}, \qquad
\omega_{p2} = \frac{1}{R_2C_2} = 5\times10^7~\text{rad/s}
\]
\[
f_{p1} = 1.59~\text{MHz}, \qquad f_{p2} = 7.96~\text{MHz}
\]
\section{Lecture 6}
\subsection{Miller compensation - Bode Plot}
Draw Bode plot for miller compensated circuit. \\ $G_{m_1} = G_{m_2} = 10 \mu S$ , $R_1 = R_2 = 10M \Omega$ , $C_1 = 10fF, C_2 = 2fF$ \\ Find Phase Margin, Unity Gain Bandwidth and DC Gain. (Theory + Cadence) (AC analysis/STB)
\begin{figure}[H]
    \centering
    \includegraphics[width=0.8\linewidth]{images/Lec_6_Q_1.jpeg}
    \caption{Lecture 6 - Question 1}
\end{figure}
\subsection{Adding a zero to a system}
Simulate in cadence.
\begin{figure}[H]
    \centering
    \includegraphics[width=0.8\linewidth]{images/Lec_6_Q_2.jpeg}
    \caption{Lecture 6 - Question 2}
\end{figure}
\section{Lecture 7}
\subsection{\colorbox{yellow}{Characterization of Resistors in TSMC 180nm}}
% There were 3 types of resistors in general. 
% Poly resistors 
% N well or p well resistors
% Metal resistors
% Choose few of them randomly and try these
TSMC 18 lib, Simulate all the resistors $-40^\circ$ to $125^\circ$ (SS, TT, FF)
TSMC 18 lib, Simulate all the resistors $-40^\circ$ to $125^\circ$ (SS, TT, FF)
\begin{table}[ht]
\centering
\begin{tabular}{|l|l|l|l|l|l|}
\hline
Resistor & Temperature Coefficient & Type & Process Spread & Linearity & Area \\
\hline
R1 & Value & TypeA & Spread1 & Linear1 & Area1 \\
R2 & Value & TypeB & Spread2 & Linear2 & Area2 \\
R3 & Value & TypeC & Spread3 & Linear3 & Area3 \\
\hline
\end{tabular}
\caption{Resistor characteristics under different conditions}
\end{table}
\begin{figure}[H]
    \centering
    \includegraphics[width=0.8\linewidth]{images/Lec_7_Q_1_2_3.jpeg}
    \caption{Lecture 7 - Question 1,2,3}
\end{figure}
\begin{figure}
    \centering
    \includegraphics[width=1\linewidth]{images/Lec_7_Q_1_ckt.png}
    \caption{Lec 7 : Question 1: Testbench}
\end{figure}
All the resistors are adjusted to a resistance of 1k\(\Omega\).

\begin{table}[H]
\centering
\caption{Comparison of Different Resistors}
% Reduce font slightly and scale the table to the text width so it fits page margins
{\scriptsize
\resizebox{\textwidth}{!}{%
\begin{tabular}{|c|c|c|c|c|c|}
\hline
	extbf{Resistor} & \textbf{TC (ppm/\(^\circ\)C)} & \textbf{Type} & \textbf{Process Spread (\%)} & \textbf{Linearity} & \textbf{Area (\(\mu\)m\(^2\))} \\ \hline
res (analogLib)   & 0                     & Ideal resistor &                              & 0             &                           \\ \hline
rphpoly           & -297.56               & p+ polysilicon resistor       & 4.9            & 0.28          & 27.116      \\ \hline
rphpoly\_dis      & -297.56               & p+ polysilicon                & 4.9            & 0.28          & 27.116      \\ \hline
rphpoly\_rf       & -270.033              & p+ polysilicon RF resistor    & 4.4            & 0.27          & 12.6        \\ \hline
rnhpoly           & -1375.19              & n+ polysilicon resistor       & 22.68          & 0.81          & 24.7084     \\ \hline
rnwell            & 3324.13               & n-well resistor               & 54.8           & 3.31          & 50.37       \\ \hline
\end{tabular}%
} % end resizebox
} % end scriptsize group
\end{table}
\subsection{Characterization of Capacitors in TSMC 180nm}
Repeate Question 1 with a capacitor. 
\begin{table}[ht]
\centering
\begin{tabular}{|l|l|l|l|l|l|}
\hline
Capacitor & Temperature Coefficient & Type & Process Spread & Linearity & Area \\
\hline
C1 & Value & TypeA & Spread1 & Linear1 & Area1 \\
C2 & Value & TypeB & Spread2 & Linear2 & Area2 \\
C3 & Value & TypeC & Spread3 & Linear3 & Area3 \\
\hline
\end{tabular}
\caption{Resistor characteristics under different conditions}
\end{table}

\subsection{\colorbox{yellow}{PTAT, CTAT, Ref - Voltage, Current and Resistance Design Combinations}}
$V_{PTAT}, V_{CTAT}, V_{Ref}$\\ $I_{PTAT}, I_{CTAT}, I_{Ref}$\\ $R_{PTAT}, R_{CTAT}, R_{Ref}$\\
All P and C for 9 values. 
\subsubsection*{Solution:}
\textbf{Formulas for $V_{PTAT}$:}
\begin{gather*}
V_{PTAT} = I_{PTAT} \cdot R_{Ref} \\
V_{PTAT} = I_{Ref} \cdot R_{PTAT} \\
V_{PTAT} = \frac{V_{CTAT}}{R_{CTAT}}\cdot R_{PTAT} \\
V_{PTAT} = \frac{V_{PTAT}}{R_{PTAT}}\cdot R_{PTAT} \\
V_{PTAT} = (I_{PTAT} + I_{CTAT}) \cdot R_{PTAT}\\
V_{PTAT} = V_{Ref} - V_{CTAT} \\
V_{PTAT} = I_{CTAT}\text{(small slope)} \cdot R_{PTAT} \text{ (Non-linear)}
\end{gather*}

\textbf{Formulas for $V_{CTAT}$:}
\begin{gather*}
V_{CTAT} = I_{CTAT} \cdot R_{Ref} \\
V_{CTAT} = I_{Ref} \cdot R_{CTAT} \\
V_{CTAT} = \frac{V_{PTAT}}{R_{PTAT}}\cdot R_{CTAT} \\
V_{CTAT} = \frac{V_{CTAT}}{R_{CTAT}}\cdot R_{CTAT} \\
V_{CTAT} = (I_{PTAT} + I_{CTAT}) \cdot R_{CTAT} \\
V_{CTAT} = V_{Ref} - V_{PTAT} \\
V_{CTAT} = I_{PTAT}\text{(small slope)} \cdot R_{CTAT} \text{ (Non-linear)}
\end{gather*}

\textbf{Formulas for $V_{Ref}$:}
\begin{gather*}
V_{Ref} = I_{Ref} \cdot R_{Ref} \\
V_{Ref} = (I_{PTAT} + I_{CTAT}) \cdot R_{Ref} \\
V_{Ref} = \frac{V_{PTAT}}{R_{PTAT}}\cdot R_{Ref} \\
V_{Ref} = \frac{V_{CTAT}}{R_{CTAT}}\cdot R_{Ref} \\
V_{Ref} = \frac{V_{PTAT} + V_{CTAT}}{R_{Ref}}\cdot R_{Ref} \\
V_{Ref} = V_{PTAT_1} - V_{PTAT_2} \\
V_{Ref} = V_{CTAT_1} - V_{CTAT_2}
\end{gather*}

\textbf{Formulas for $I_{PTAT}$, $I_{CTAT}$, $I_{Ref}$:}
\begin{gather*}
I_{PTAT} = \frac{V_{PTAT}}{R_{Ref}} \\
I_{PTAT} = \frac{I_{PTAT} \cdot R_{PTAT}}{R_{Ref}} \\
I_{CTAT} = \frac{V_{CTAT}}{R_{Ref}} \\
I_{CTAT} = \frac{I_{CTAT} \cdot R_{CTAT}}{R_{Ref}} \\
I_{CTAT} = \frac{V_{CTAT}}{R_{CTAT}} \\
I_{Ref} = \frac{V_{Ref}}{R_{Ref}}\\
I_{Ref} = \frac{V_{CTAT}}{R_{CTAT}}\\
I_{Ref} = \frac{V_{PTAT}}{R_{PTAT}}\\
I_{Ref} = \frac{V_{PTAT} + V_{CTAT}}{R_{Ref}}\\
I_{Ref}= \frac{I_{CTAT} \cdot R_{Ref}}{R_{CTAT}}\\
I_{Ref}= \frac{I_{PTAT} \cdot R_{Ref}}{R_{PTAT}}
\end{gather*}

\textbf{Formulas for $R_{PTAT}$, $R_{CTAT}$, $R_{Ref}$:}
\begin{gather*}
R_{PTAT} = \frac{V_{PTAT}}{I_{Ref}} \\
R_{PTAT} = \frac{V_{Ref} - V_{CTAT}}{I_{Ref}} \\
R_{PTAT} = \frac{V_{PTAT} \cdot R_{PTAT}}{V_{CTAT}}\\
R_{PTAT} = \frac{V_{PTAT} \cdot R_{CTAT}}{V_{CTAT}}\\
R_{CTAT} = \frac{V_{CTAT}}{I_{Ref}} \\
R_{CTAT} = \frac{V_{Ref} - V_{PTAT}}{I_{Ref}} \\
R_{Ref} = \frac{V_{Ref}}{I_{Ref}}\\
R_{CTAT} = \frac{V_{CTAT} \cdot R_{PTAT}}{V_{PTAT}}\\
R_{CTAT} = \frac{V_{CTAT} \cdot R_{CTAT}}{V_{CTAT}}\\
R_{Ref} = \frac{V_{PTAT}}{I_{PTAT}}
\end{gather*}

\section{Lecture 8 - Part of MidSem Project}
\begin{figure}[H]
    \centering
    \includegraphics[width=0.8\linewidth]{images/Lec_8_Q1_ckt.jpeg}
    \caption{Lecture 8 - Question 1: Circuit}
\end{figure}
\textbf{Answer: }\\

\begin{figure}[H]
    \centering
    \includegraphics[width=1\linewidth]{images/Lec_8_Q1.jpeg}
    \caption{Lec 8: Question 1}
\end{figure}
\subsection{\colorbox{yellow}{Slope of $V_{EB}$ vs Temperature curve of a BJT}}
\begin{figure}[H]
    \centering
    \includegraphics[width=0.8\linewidth]{images/Lec_8_Q_1_ckt.png}
    \caption{Lecture 8 - Question 1 - Testbench}
\end{figure}

\begin{figure}[H]
    \centering
    \includegraphics[width=1\linewidth]{images/Lec_8_Q_1_V_EB_graphs.png}
    \caption{Lec 8: Q1: $V_{EB}$ vs Temperature curve of a BJT}
\end{figure}

\begin{figure}[H]
    \centering
    \includegraphics[width=1\linewidth]{images/Lec_8_Q_1_V_EB_slope_graphs.png}
    \caption{Lec 8: Q1: Slope Calculation of $V_{EB}$ vs Temperature curve of a BJT}
\end{figure}
\subsection{Difference of $V_{EB}$ of two BJTs}
\subsection{\colorbox{yellow}{Find the multiplication factor to equate PTAT and CTAT voltages}}
At $27^\circ$C, the measured values are:
\[
V_{BE1} = 634.2644 \,\text{mV}, \qquad 
\Delta V_{BE} = 54.42 \,\text{mV}
\]

The equality condition is given by:
\[
N \cdot \Delta V_{BE} = M \cdot V_{BE1}
\]

Rearranging,
\[
\frac{M}{N} = \frac{\Delta V_{BE}}{V_{BE1}}
\]
Substituting the values,
\[
\frac{M}{N} = \frac{54.42}{634.2644} \approx 0.0858
\]

Therefore,
\[
\frac{M}{N} \approx 0.086 \quad \Rightarrow \quad \frac{R_2}{R_1} \approx 0.086
\]

\subsection{Simulate the BGR using ideal VCVS, VCVS for current mirror}
\section{Lecture 9 - MOSFET}
\subsection{\colorbox{yellow}{NMOS I-V characteristics}}
\begin{figure}[H]
    \centering
    \includegraphics[width=0.8\linewidth]{images/Lec_9_Q1.jpeg}
    \caption{Lecture 9 - Question 1}
\end{figure}
\textbf{Vg: 0 to 1.8V}

\begin{enumerate}[(i)]
    \item \textbf{W/L = 220n/180n}
    \begin{enumerate}[(a)]
        \item Vds = 500 mV
        \begin{figure}[H]
            \centering
            \includegraphics[width=1\linewidth]{images/tyrjrur5et.png}
            \label{fig:id_vgs_220n_500mV}
        \end{figure}
        
        \item Vds = 1.8 V
        \begin{figure}[H]
            \centering
            % \includegraphics[width=1\linewidth]{images/fvbgf.png}   \label{fig:placeholder}
        \end{figure}
    \end{enumerate}

    \item \textbf{W/L = 1µ/1µ}
    \begin{enumerate}[(a)]
        \item Vds = 500 mV
        \begin{figure}[H]
            \centering
            \includegraphics[width=1\linewidth]{images/htjy.png}
            % \label{fig:placeholder}
        \end{figure}
        \vspace{10mm}
        \item Vds = 1.8 V
        \begin{figure}[H]
            \centering
            \includegraphics[width=1\linewidth]{images/grtgr.png}
            % \label{fig:placeholder}
        \end{figure}
    \end{enumerate}
\end{enumerate}

\textbf{Vd: 0 to 1.8V}

\begin{enumerate}[(i)]
    \item \textbf{W/L = 220n/180n}
    \begin{enumerate}[(a)]
        \item Vgs = 500 mV
        \begin{figure}[H]
            \centering
            \includegraphics[width=1\linewidth]{images/fvbcn.png}
            % \label{fig:placeholder}
        \end{figure}
        \vspace{10mm}
        \item Vgs = 1.8 V
        \begin{figure}[H]
            \centering
            \includegraphics[width=1\linewidth]{images/fgb.png}
            % \label{fig:placeholder}
        \end{figure}
    \end{enumerate}

    \item \textbf{W/L = 1µ/1µ}
    \begin{enumerate}[(a)]
        \item Vgs = 500 mV
        \begin{figure}[H]
            \centering
            \includegraphics[width=1\linewidth]{images/grtu7i.png}
            % \label{fig:placeholder}
        \end{figure}
        \vspace{10mm}
        \item Vgs = 1.8 V
        \begin{figure}[H]
            \centering
            \includegraphics[width=1\linewidth]{images/gthy.png}
            % \label{fig:placeholder}
        \end{figure}
    \end{enumerate}
\end{enumerate}

\subsection{\colorbox{yellow}{PMOS I-V characteristics}}
\begin{figure}[H]
    \centering
    \includegraphics[width=0.8\linewidth]{images/Lec_9_Q2.jpeg}
    \caption{Lecture 9 - Question 2}
\end{figure}

\textbf{Vg: 0 to 1.8V}

\begin{enumerate}[(i)]
    \item \textbf{W/L = 220n/180n}
    \begin{enumerate}[(a)]
        \item Vd = 500 mV   
        \begin{figure}[H]
            \centering
            \includegraphics[width=1\linewidth]{images/u7g.png}
            % \label{fig:placeholder}
        \end{figure}
        \vspace{10mm}
        \item Vd = 1.8 V
        \begin{figure}[H]
            \centering
            \includegraphics[width=1\linewidth]{images/yj.png}
            % \label{fig:placeholder}
        \end{figure}
    \end{enumerate}

    \item \textbf{W/L = 1µ/1µ}
    \begin{enumerate}[(a)]
        \item Vd = 500 mV
        \begin{figure}[H]
            \centering
            \includegraphics[width=1\linewidth]{images/ghj.png}
            % \label{fig:placeholder}
        \end{figure}
        \vspace{10mm}
        \item Vd = 1.8 V
        \begin{figure}[H]
            \centering
            \includegraphics[width=1\linewidth]{images/dsvf.png}
            % \label{fig:placeholder}
        \end{figure}
    \end{enumerate}
\end{enumerate}

\textbf{Vd: 0 to 1.8V}

\begin{enumerate}[(i)]
    \item \textbf{W/L = 220n/180n}
    \begin{enumerate}[(a)]
        \item Vgs = 500 mV
        \begin{figure}[H]
            \centering
            \includegraphics[width=1\linewidth]{images/tyh.png}
            % \label{fig:placeholder}
        \end{figure}
        \vspace{10mm}
        \item Vgs = 1.8 V
        \begin{figure}[H]
            \centering
            \includegraphics[width=1\linewidth]{images/grthy.png}
            % \label{fig:placeholder}
        \end{figure}
    \end{enumerate}

    \item \textbf{W/L = 1µ/1µ}
    \begin{enumerate}[(a)]
        \item Vgs = 500 mV
        \begin{figure}[H]
            \centering
            \includegraphics[width=1\linewidth]{images/grgnfhmj.png}
            % \label{fig:placeholder}
        \end{figure}
        \vspace{10mm}
        \item Vgs = 1.8 V
        \begin{figure}[H]
            \centering
            \includegraphics[width=1\linewidth]{images/rgfrg.png}
            % \label{fig:placeholder}
        \end{figure}
    \end{enumerate}
\end{enumerate}

\section{Lecture 10 - Current Mirrors}
\subsection{\colorbox{yellow}{Prove that Threshold voltage is linear and CTAT}}
\begin{figure}[H]
    \centering
    \includegraphics[width=0.8\linewidth]{images/Lec_10_Q1.jpeg}
    \caption{Lecture 10 - Question 1}
\end{figure}
\textbf{Answer: }\\
\begin{figure}[H]
    \centering
    \includegraphics[width=0.75\linewidth]{images/sfegrg.png}
    \caption{Schematic}
    % \label{fig:placeholder}
\end{figure}

\begin{figure}[H]
    \centering
    \includegraphics[width=0.75\linewidth]{images/10.1.png}
    \caption{Plot of Vth vs Temperature}
    % \label{fig:placeholder}
\end{figure}

Clearly, Vth has a CTAT nature and is linear when varied with temperature.

\begin{figure}[H]
    \centering
    \includegraphics[width=1\linewidth]{images/10.2.png}
    \caption{Slope of Vth vs Temperature}
    % \label{fig:placeholder}
\end{figure}

From the above plot, we can see that the slope of the Vth is almost constant.


\begin{figure}
    \centering
    \includegraphics[width=1\linewidth]{images/10.3.png}
    \caption{Slope across slow, typical and fast corners}
    % \label{fig:placeholder}
\end{figure}

\subsection{\colorbox{yellow}{Finding the current and sizing of the mosfet required to make $V_{GS}$ as Reference}}
\begin{figure}[H]
    \centering
    \includegraphics[width=0.8\linewidth]{images/Lec_10_Q2.jpeg}
    \caption{Lecture 10 - Question 2}
\end{figure}


\begin{figure}[H]
    \centering
    \includegraphics[width=0.75\linewidth]{images/efwrvrte.png}
    \caption{Vgs vs Temperature}
    % \label{fig:placeholder}
\end{figure}

(i) Find I such that Vgs becomes Vref \\

Since Vref isn't specifically given, let's aim for Vref = 0.5V. And the W/L for this is 2u/180n. Now, we sweep the current in the circuit to find the value of I for which we obtain Vgs = Vref.

\begin{figure}[H]
    \centering
    \includegraphics[width=0.75\linewidth]{images/Vgs vs Id.png}
    \caption{Vgs vs Id}
    % \label{fig:placeholder}
\end{figure}

I = 5.75 uA for Vgs = Vref = 0.5V \\


\noindent
(ii) Find W/L such that Vgs becomes Vref \\
Let us consider Vref = 0.5V and reference current as 3uA.\\

Let L = 180n and W = k * 180n


\begin{figure}[H]
    \centering
    \includegraphics[width=0.75\linewidth]{images/srbreberg.png}
    \caption{Vgs vs k}
    % \label{fig:placeholder}
\end{figure}

From the sweep of value of k, we obtain Vref at Vgs at k = 25.05.

Thus, W/L = 25.05.

\subsection{\colorbox{yellow}{PMOS Current Mirror - Simple, Widlar, Low swing, High swing}}
\begin{figure}[H]
    \centering
    \includegraphics[width=0.8\linewidth]{images/Lec_10_Q3.jpeg}
    \caption{Lecture 10 - Question 3}
\end{figure}
\textbf{Answer: }\\

\begin{figure}[H]
    \centering
    \includegraphics[width=1\linewidth]{images/Lec_10_Q3_Soln.jpg}
    \caption{Lecture 10 - Question 3 - Solution}
\end{figure}

\begin{figure}[H]
    \centering
    \includegraphics[width=1\linewidth]{images/Lec_10_Q3_Soln_2.jpg}
    \caption{Lecture 10 - Question 3 - Plots}
\end{figure}

\subsection{\colorbox{yellow}{Plot $V_{EB}$ w.r.t Process}}
\begin{figure}[H]
    \centering
    \includegraphics[width=0.8\linewidth]{images/Lec_10_Q4.jpeg}
    \caption{Lecture 10 - Question 4}
\end{figure}
\begin{figure}[H]
    \centering
    \includegraphics[width=1\linewidth]{images/ryjtuk.png}
    \caption{Vbe in different corners}
    % \label{fig:placeholder}
\end{figure}

\section{Lecture 11 - Startup Condition, Power Down Signal, Random Mismatch}
\begin{figure}[H]
    \centering
    \includegraphics[width=0.8\linewidth]{images/Lec_11_Questions.jpeg}
    \caption{Lecture 11 - Question 1 and Question 2}
\end{figure}
\subsection{\colorbox{yellow}{Question 1}}
\begin{figure}[H]
    \centering
    \includegraphics[width=1\linewidth]{images/11.3.png}
    % \label{fig:placeholder}
\end{figure}

\begin{figure}[H]
    \centering
    \includegraphics[width=1\linewidth]{images/ythhh.png}
    % \label{fig:placeholder}
\end{figure}

(i) Plot I vs t without Startup circuit
\begin{figure}[H]
    \centering
    \includegraphics[width=1\linewidth]{images/utjtutu.png}
    \caption{Current in the circuit without startup circuit}
    % \label{fig:placeholder}
\end{figure}

(ii) Plot I vs t with Startup circuit
\begin{figure}[H]
    \centering
    \includegraphics[width=1\linewidth]{images/11.1.png}
    % \label{fig:placeholder}
\end{figure}

We can see that without the startup circuit, the current in the brach has settled to the desired 0.67uA after significant time of 80us. \\
But, after introducing the startup circuit, we can see that the current has settled in afew nano seconds. This verifies that the startup circuit is working. \\

(iii) Current in Startup Circuit
\begin{figure}[H]
    \centering
    \includegraphics[width=1\linewidth]{images/asfefreg.png}
    \caption{Current in startup circuit}
    % \label{fig:placeholder}
\end{figure}

The current in the startup circuit is almost 3pA after the current in the branches is settled at required 0.67uA, which is very small as needed, since it should interfere with the normal working of the circuit. 

\subsection{\colorbox{yellow}{Question 2}}
\begin{figure}[H]
    \centering
    \includegraphics[width=1\linewidth]{images/11.2.png}
    % \label{fig:placeholder}
\end{figure}

(i) 220n/180n
\begin{figure}[H]
    \centering
    \includegraphics[width=0.75\linewidth]{images/jyjyjk.png}
    % \label{fig:placeholder}
\end{figure}

\noindent
Mean =2.657u\\
Std Dev = 42.72n\\

(ii) 10u/10u
\begin{figure}[H]
    \centering
    \includegraphics[width=0.75\linewidth]{images/sfbgberhe.png}
    % \label{fig:placeholder}
\end{figure}

\noindent
Mean = 1.026u\\
Std Dev = 5.628n\\

We can clearly observe that smaller device dimensions result in significantly higher mismatch, whereas larger device dimensions exhibit much lower mismatch.

This behavior is explained by Pelgrom's mismatch model, which states that the standard deviation of mismatch is inversely proportional to the square root of the device area:

\[
\sigma_{\text{mismatch}} \propto \frac{1}{\sqrt{WL}}
\]

Thus, increasing the transistor width \(W\) and length \(L\) decreases the mismatch variation.

\section{Lecture 12 - Small Signal Analysis of Current Mirrors}
\subsection{\colorbox{yellow}{Derivation of small signal equivalent}}
\begin{figure}[H]
    \centering
    \includegraphics[width=0.8\linewidth]{images/Lec_12_Q1.jpeg}
    \caption{Lecture 12 - Question 1}
\end{figure}
\subsubsection*{Solution:}
\begin{figure}[H]
    \centering
    \includegraphics[width=1\linewidth]{images/Lec_12_Q1_Soln.jpg}
    \caption{Lecture 12 - Question 1 - Solution}
\end{figure}

\subsection{Finding the output resistance of the current mirror}
\begin{figure}[H]
    \centering
    \includegraphics[width=0.8\linewidth]{images/Lec_12_Q2.jpeg}
    \caption{Lecture 12 - Question 2}
\end{figure}

\section{Lecture 13 - Amplifiers}
\subsection{\colorbox{green}{CS Amplifier with resistive load}}
Design a CS amplifier with resistive load to get a gain of 20dB, $V_{DD} = 1.8V$, $I_Q$ = 1$\mu A$
\subsubsection*{Solution:}
\begin{figure}[H]
    \centering
    \includegraphics[width=1\linewidth]{images/Lec_13_Q_Soln_theory.jpg}
    \caption{Lecture 13 - Question 1 - Solution - Theory}
\end{figure}

\section{Lecture 14 - Single Stage Amplifiers}
\subsection{\colorbox{yellow}{Drain feedback CS Amplifier}}
Find conditions for $C_1$, $C_2$ to be short circuit in a drain feedback CS amplifier(NMOS based).
\begin{figure}[H]
    \centering
    \includegraphics[width=0.8\linewidth]{images/Lec_14_Q1.jpg}
    \caption{Lecture 14 - Question 1}
\end{figure}

\subsection{\colorbox{yellow}{Source feedback CS Amplifier}}
Find conditions on $R_G$, $C_1$, $C_2$ in source feedback bias CS Amplifier(NMOS based).
\begin{figure}[H]
    \centering
    \includegraphics[width=0.8\linewidth]{images/Lec_14_Q2.jpg}
    \caption{Lecture 14 - Question 2}
\end{figure}

\subsection{\colorbox{yellow}{PMOS equivalent of CS Amplifiers}}
Draw PMOS equivalent of voltage mode CS Amplifier (6), Current Mode CS Amplifier(3), CD and CG amplifier.
\subsubsection*{Solution:}
\begin{figure}[H]
    \centering
    \includegraphics[width=1\linewidth]{images/Lec_14_Q3_Soln_part_1.jpg}
    \caption{Lecture 14 - Question 3 - Solution - Part 1}
\end{figure}

\begin{figure}[H]
    \centering
    \includegraphics[width=1\linewidth]{images/Lec_14_Q3_Soln_part_2.jpg}
    \caption{Lecture 14 - Question 3 - Solution - Part 2}
\end{figure}

\section{\colorbox{yellow}{Lecture 15 - Differential Amplifiers}}
\subsection{\colorbox{yellow}{ICMR and OCMR of PMOS differential pair}}
\begin{figure}[H]
    \centering
    \includegraphics[width=0.8\linewidth]{images/Lec_15_Q.png}
    \caption{Lecture 15 - Question 1}
\end{figure}
\subsubsection*{Solution:}
\begin{figure}[H]
    \centering
    \includegraphics[width=1\linewidth]{images/Lec_15_Q_Soln.jpg}
    \caption{Lecture 15 - Question 1 - Solution}
\end{figure}

\section{Lecture 16 - Parasitic Capacitances and Poles}
\subsection{\colorbox{yellow}{Bode Plot of NMOS CS Amp with capacitive load}}
\begin{figure}[H]
    \centering
    \includegraphics[width=0.8\linewidth]{images/Lec_16_Q.png}
    \caption{Lecture 16 - Question 1}
\end{figure}
\subsubsection*{Solution:}
\begin{figure}[H]
    \centering
    \includegraphics[width=1\linewidth]{images/Lec_16_Q_Soln.jpg}
    \caption{Lecture 16 - Question 1 - Solution}
\end{figure}

\section{Lecture 17}
\end{document}