\documentclass[a4paper]{article}

%% Language and font encodings
\usepackage[english]{babel}
\usepackage[utf8x]{inputenc}
\usepackage[T1]{fontenc}

%% Page size and margins
\usepackage[a4paper,top=3cm,bottom=2cm,left=3cm,right=3cm,marginparwidth=1.75cm]{geometry}

%% Packages
\usepackage{amsmath}
\usepackage{bm}
\usepackage{graphicx}
\usepackage{algorithm}
\usepackage{float}
\usepackage{caption}
\usepackage{subcaption}
\usepackage{enumerate}
\usepackage[noend]
{algpseudocode}
\setlength{\marginparwidth}{2cm}
\usepackage[colorinlistoftodos]{todonotes}
\usepackage[colorlinks=false, allcolors=blue]{hyperref}
\usepackage{textcomp}
\usepackage{xcolor}


\title{\textbf{CMOS References and Regulators: Homework}}
\author{Chamarthy Madhan Sai Krishna\\2023102030}

\begin{document}

\maketitle

\tableofcontents

\section{Lecture 1}

\subsection{\colorbox{yellow}{Proving Resistance: Current is proportional to Voltage}}
Do the 'DC Analysis' for the following circuits. Plot V vs I $\&$ write expression for the slope. Prove that its a resistance.\\
\begin{figure}[H]
    \centering
    \includegraphics[width=0.8\linewidth]{images/Lec_1_Q_1.jpeg}
    \caption{Lecture 1 - Question 1 - (a) and (b)}
\end{figure}
\textit{Answer: }\\
Left graph shows the I - V characteristics of the device. Clearly, $I \propto V$. The right graph shows the expression for the slope. \\
Plot for (a) : Using a Voltage source \\
Plot for (b) : Using a Current source

\begin{figure}[H]
    \centering
    \includegraphics[width=1\linewidth]{images/Lec_1_Q_1_testbench.png}
    \caption{Lec 1: Q1: Testbench for (a) and (b)}
\end{figure}

\begin{figure}[H]
    \centering
    \includegraphics[width=1\linewidth]{images/Lec_1_Q_1_a_plot.png}
    \caption{Lec 1: Q1: Plot for (a)}
\end{figure}

\begin{figure}[H]
    \centering
    \includegraphics[width=1\linewidth]{images/Lec_1_Q_1_b_plot.png}
    \caption{Lec 1: Q1: Plot for (b)}
\end{figure}

\subsection{\colorbox{yellow}{Capacitance Amplifier using Controlled Sources}}
Design a capacitance amplifier using controlled sources.\\
\begin{figure}[H]
    \centering
    \includegraphics[width=0.8\linewidth]{images/Lec_1_Q_2_3_4.jpeg}
    \caption{Lecture 1 - Question 2, 3, 4}
\end{figure}
\textit{Answer: }\\
\begin{figure}[H]
    \centering
    \includegraphics[width=1\linewidth]{images/Lec_1_Q2_Soln.jpeg}
    \caption{Lecture 1 - Question 2 - Solution}
\end{figure}
\subsection{\colorbox{yellow}{Inductance Amplifier using Controlled Sources}}
Design a inductance amplifier using controlled sources.\\

\textit{Answer: }\\

\begin{figure}[H]
    \centering
    \includegraphics[width=1\linewidth]{images/Lec_1_Q3_Soln.jpeg}
    \caption{Lecture 1 - Question 3 - Solution}
\end{figure}

\subsection{\colorbox{yellow}{Resistance Amplifier using Controlled Sources}}
Sweep V vs I for (i) $k>1$ (ii) $0<k<1$ (iii) $K<0$ Prove $R_{eff}$ in each case is $\frac{R}{1-k}$.\\
\textit{Answer: }\\
\begin{figure}[H]
    \centering
    \includegraphics[width=1\linewidth]{images/Lec_1_Q4_Soln.jpeg}
    \caption{Lecture 1 - Question 4 - Solution}
\end{figure}

\section{Lecture 2}
\subsection{\colorbox{yellow}{Output Resistance}}
find the effective output resistance at $V_{out}$ i.e., $R_{out} = ?$ \\
\begin{figure}[H]
    \centering
    \includegraphics[width=1\linewidth]{images/Lec_2_Q_1_Soln.jpeg}
    \caption{Lec 2: Question 1}
\end{figure}

\begin{figure}[H]
    \centering
    \includegraphics[width=0.8\linewidth]{images/Lec_2_Q1.jpeg}
    \caption{Lecture 2 - Question 1 - Solution}
\end{figure}
\textit{Answer:}\\
When we nullify the V1, the vccs vanish and the effective output resistance at vout node is $R_{out} = r_{ds_1} || r_{ds_2}$. 

Proved this using the plot with values.

\begin{figure}[H]
    \centering
    \includegraphics[width=1\linewidth]{images/Lec_2_Q_1_testbench.png}
    \caption{Lec 2: Q1: Testbench}
\end{figure}

\subsection{\colorbox{yellow}{Step Response for RC Circuit}}
Plot $V_{out}$ vs time in Cadence. Prove Theory. \\
\begin{figure}[H]
    \centering
    \includegraphics[width=0.8\linewidth]{images/Lec_2_Q_2.jpeg}
    \caption{Lecture 2 - Question 2}
\end{figure}
\textit{Answer:}\\
Copy this from the notes\\
\begin{figure}[H]
    \centering
    \includegraphics[width=1\linewidth]{images/Lec_2_Q2_Soln_Part_1.jpeg}
    \caption{Lec 2: Solution for Question 2}
\end{figure}

\begin{figure}[H]
    \centering
    \includegraphics[width=1\linewidth]{images/Lec_2_Q2_Soln_Part_2.jpeg}
    \caption{Lec 2: Solution for Question 2}
\end{figure}

\begin{figure}[H]
    \centering
    \includegraphics[width=1\linewidth]{images/Lec_2_Q2_Soln_Part_3.jpeg}
    \caption{Lec 2: Solution for Question 2}
\end{figure}
Proved this using theory and from the simulations  by finding the output voltage after 1 time constant.

\begin{figure}[H]
    \centering
    \includegraphics[width=1\linewidth]{images/Lec_2_Q_2_testbench.png}
    \caption{Lec 2: uestion 2 Testbench}
\end{figure}

\begin{figure}[H]
    \centering
    \includegraphics[width=1\linewidth]{images/Lec_2_Q_2_a.png}
    \caption{Lec 2: Q2: Soln - Part (a)}
\end{figure}

\begin{figure}[H]
    \centering
    \includegraphics[width=1\linewidth]{images/Lec_2_Q_2_b.png}
    \caption{Lec 2 : Q2: Soln - Part (b)}
\end{figure}

\section{Lecture 3}
\subsection{\colorbox{yellow}{Step Response for passive RLC circuit Combinations}}
Plot $V_{out}$ vs $time$ and $I_{out}$ vs $time$
\begin{itemize}
    \item 
    \item 
    \item 
\end{itemize}
\begin{figure}[H]
    \centering
    \includegraphics[width=0.8\linewidth]{images/Lec_3_Q_1.jpeg}
    \caption{Lecture 3 - Question 1}
\end{figure}

\begin{figure}[H]
    \centering
    \includegraphics[width=1\linewidth]{images/Lec_3_Q1_Soln_1.jpeg}
    \caption{Lec 3: Solution for Question 1}
\end{figure}

\begin{figure}[H]
    \centering
    \includegraphics[width=1\linewidth]{images/Lec_3_Q1_Soln_2.jpeg}
    \caption{Lec 3: Solution for Question 1}
\end{figure}


\subsection{\colorbox{yellow}{Pulse Response for RC Circuit with varying Duty Cycle and Time Period}}
Plot $V_{out}$ vs $time$ (a) $D = 10\%$ (b) $D = 90\%$ (i) $T = 0.1RC$ (ii) $T = 10RC$ where $D = \frac{T_{ON}}{T}$
\begin{figure}[H]
    \centering
    \includegraphics[width=0.8\linewidth]{images/Lec_3_Q_2.jpeg}
    \caption{Lecture 3 - Question 2}
\end{figure}

\begin{figure}[H]
    \centering
    \includegraphics[width=1\linewidth]{images/Lec_3_Q2_Soln.jpeg}
    \caption{Lec 3: Solution for Question 2}
\end{figure}
\subsection{\colorbox{yellow}{Bode Plot for a given Transfer Function}}
Draw Bode plot for $H(s) = \frac{(s - z_1)(s+z_2)}{(s+P_1)(s+P_2)(s+P_3)}$ where (i) $z_2 << p_1 << p_2 << p_3 << z_1$ (ii) $p_3 << p_2 << p_1 << z_1 << z_2$
\begin{figure}[H]
    \centering
    \includegraphics[width=0.8\linewidth]{images/Lec_3_Q_3.jpeg}
    \caption{Lecture 3 - Question 3}
\end{figure}
\begin{figure}[H]
    \centering
    \includegraphics[width=1\linewidth]{images/Lec_3_Q3_Soln.jpeg}
    \caption{Lec 3: Solution for Question 3}
\end{figure}
\section{Lecture 4}
\subsection{\colorbox{yellow}{Non-inverting Schimtt Trigger}}
Design a non-inverting schimtt trigger using opamp based circuits.
\begin{figure}[H]
    \centering
    \includegraphics[width=0.8\linewidth]{images/Lec_4_Q_1.jpeg}
    \caption{Lecture 4 - Question 1}
\end{figure}
\begin{figure}[H]
    \centering
    \includegraphics[width=1\linewidth]{images/Lec_4_Q1_Soln.jpeg}
    \caption{Lec 4: Solution for Question 1}
\end{figure}
\subsection{\colorbox{green}{Expression for Closed loop Unity gain bandwidth}}
Find the expression for Closed loop Unity gain bandwidth.
\begin{figure}[H]
    \centering
    \includegraphics[width=0.8\linewidth]{images/Lec_4_Q_2_3.jpeg}
    \caption{Lecture 4 - Question 2,3}
\end{figure}
\textbf{Answer:}\\
Assuming a single pole system, the below image shows the closed loop unity gain bandwidth.\\
\begin{figure}[H]
    \centering
    \includegraphics[width=1\linewidth]{images/Lec_4_Q2_Soln.jpeg}
    \caption{Lec 4: Solution for Question 2}
\end{figure}
\subsection{\colorbox{yellow}{Non-Inverting Amplifier using VCVS}}
In cadence, Plot $V_o$ vs $V_{in}$ (Sweep $V_{in}$ from $-V_{ss}$ to $V_{DD}$) for non-inverting amplifier using Opamp and VCVS for both positive and -ve feedback.

\textbf{Answer:}\\
A non--inverting amplifier using postivie feedback is not possible. Due to positive feedback, the output will saturate to either $V_{DD}$ or $V_{SS}$ for any non-zero input.\\
For negative feedback, the output voltage is given by\\ $V_{out} = V_{in} \cdot (1 + \frac{R_2}{R_1})$.\\
For the non-inverting amplifier we can use either an opamp or a VCVS.\\
\begin{figure}[H]
    \centering
    \includegraphics[width=0.8\linewidth]{images/Lec_4_Q3_Soln.png}
    \caption{Lecture 4 - Question 3 Solution: Non-inverting Amplifier using VCVS}
\end{figure}
\section{Lecture 5}
\subsection{Single Pole Opamp with Feedback - Bode Plot}
In Cadence, Bode Plot for 
\begin{enumerate}
    \item A = 100, check $V_x$ (offset) [DC]
    \item A = 1000, check $V_x$ (offset) [DC]
    \item UGB = 1MHz, Gain = 1000, Design $G_m$, $R_{out}$, $R_{1}$, $R_{2}$, $C$, $V_{Ref} = 0.6V$, $V_{out} = 1.2V$
\end{enumerate}
\begin{figure}[H]
    \centering
    \includegraphics[width=0.8\linewidth]{images/Lec_5_Q_1.jpeg}
    \caption{Lecture 5 - Question 1}
\end{figure}
\textbf{Answer:}\\
\begin{figure}[H]
    \centering
    \includegraphics[width=1\linewidth]{images/Lec_5_Q1_1_Soln.png}
    \caption{Lec 5: Solution for Question 1\_1)}
\end{figure}
\begin{figure}[H]
    \centering
    \includegraphics[width=1\linewidth]{images/Lec_5_Q1_2_Soln.png}
    \caption{Lec 5: Solution for Question 1\_2}
\end{figure}

\subsection{Bode Plot for a 2 Pole System}
In cadence, AC analysis. Bode Plot for $\frac{V_{out}}{V_{{in}}}$
\begin{enumerate}
    \item $G_{m_1} = G_{m_2} = 10 \mu F$ , $R_1 = R_2 = 10M \Omega$ , $C_1 = 10fF, C_2 = 2fF$
    \item add $C_{effective}$ at $C_1$ and get phase margin = $45^\circ$
\end{enumerate}
\begin{figure}[H]
    \centering
    \includegraphics[width=0.8\linewidth]{images/Lec_5_Q_2_ckt.jpeg}
    \caption{Lecture 5 - Question 2 - circuit}
\end{figure}
\begin{figure}[H]
    \centering
    \includegraphics[width=1\linewidth]{images/Lec_5_Q_2_ques.jpeg}
    \caption{Lecture 5 - Question 2 - Question}
\end{figure}
\section{Lecture 6}
\subsection{Miller compensation - Bode Plot}
Draw Bode plot for miller compensated circuit. \\ $G_{m_1} = G_{m_2} = 10 \mu S$ , $R_1 = R_2 = 10M \Omega$ , $C_1 = 10fF, C_2 = 2fF$ \\ Find Phase Margin, Unity Gain Bandwidth and DC Gain. (Theory + Cadence) (AC analysis/STB)
\begin{figure}[H]
    \centering
    \includegraphics[width=0.8\linewidth]{images/Lec_6_Q_1.jpeg}
    \caption{Lecture 6 - Question 1}
\end{figure}
\subsection{Adding a zero to a system}
Simulate in cadence.
\begin{figure}[H]
    \centering
    \includegraphics[width=0.8\linewidth]{images/Lec_6_Q_2.jpeg}
    \caption{Lecture 6 - Question 2}
\end{figure}
\section{Lecture 7}
\subsection{Characterization of Resistors in TSMC 180nm}
% There were 3 types of resistors in general. 
% Poly resistors 
% N well or p well resistors
% Metal resistors
% Choose few of them randomly and try these
TSMC 18 lib, Simulate all the resistors $-40^\circ$ to $125^\circ$ (SS, TT, FF)
TSMC 18 lib, Simulate all the resistors $-40^\circ$ to $125^\circ$ (SS, TT, FF)
\begin{table}[ht]
\centering
\begin{tabular}{|l|l|l|l|l|l|}
\hline
Resistor & Temperature Coefficient & Type & Process Spread & Linearity & Area \\
\hline
R1 & Value & TypeA & Spread1 & Linear1 & Area1 \\
R2 & Value & TypeB & Spread2 & Linear2 & Area2 \\
R3 & Value & TypeC & Spread3 & Linear3 & Area3 \\
\hline
\end{tabular}
\caption{Resistor characteristics under different conditions}
\end{table}
\begin{figure}[H]
    \centering
    \includegraphics[width=0.8\linewidth]{images/Lec_7_Q_1_2_3.jpeg}
    \caption{Lecture 7 - Question 1,2,3}
\end{figure}
\subsection{Characterization of Capacitors in TSMC 180nm}
Repeate Question 1 with a capacitor. 
\begin{table}[ht]
\centering
\begin{tabular}{|l|l|l|l|l|l|}
\hline
Capacitor & Temperature Coefficient & Type & Process Spread & Linearity & Area \\
\hline
C1 & Value & TypeA & Spread1 & Linear1 & Area1 \\
C2 & Value & TypeB & Spread2 & Linear2 & Area2 \\
C3 & Value & TypeC & Spread3 & Linear3 & Area3 \\
\hline
\end{tabular}
\caption{Resistor characteristics under different conditions}
\end{table}

\subsection{\colorbox{yellow}{PTAT, CTAT, Ref - Voltage, Current and Resistance Design Combinations}}
$V_{PTAT}, V_{CTAT}, V_{Ref}$\\ $I_{PTAT}, I_{CTAT}, I_{Ref}$\\ $R_{PTAT}, R_{CTAT}, R_{Ref}$\\
All P and C for 9 values. 
\subsubsection*{Solution:}
\textbf{Formulas for $V_{PTAT}$:}
\begin{gather*}
V_{PTAT} = I_{PTAT} \cdot R_{Ref} \\
V_{PTAT} = I_{Ref} \cdot R_{PTAT} \\
V_{PTAT} = \frac{V_{CTAT}}{R_{CTAT}}\cdot R_{PTAT} \\
V_{PTAT} = \frac{V_{PTAT}}{R_{PTAT}}\cdot R_{PTAT} \\
V_{PTAT} = (I_{PTAT} + I_{CTAT}) \cdot R_{PTAT}\\
V_{PTAT} = V_{Ref} - V_{CTAT} \\
V_{PTAT} = I_{CTAT}\text{(small slope)} \cdot R_{PTAT} \text{ (Non-linear)}
\end{gather*}

\textbf{Formulas for $V_{CTAT}$:}
\begin{gather*}
V_{CTAT} = I_{CTAT} \cdot R_{Ref} \\
V_{CTAT} = I_{Ref} \cdot R_{CTAT} \\
V_{CTAT} = \frac{V_{PTAT}}{R_{PTAT}}\cdot R_{CTAT} \\
V_{CTAT} = \frac{V_{CTAT}}{R_{CTAT}}\cdot R_{CTAT} \\
V_{CTAT} = (I_{PTAT} + I_{CTAT}) \cdot R_{CTAT} \\
V_{CTAT} = V_{Ref} - V_{PTAT} \\
V_{CTAT} = I_{PTAT}\text{(small slope)} \cdot R_{CTAT} \text{ (Non-linear)}
\end{gather*}

\textbf{Formulas for $V_{Ref}$:}
\begin{gather*}
V_{Ref} = I_{Ref} \cdot R_{Ref} \\
V_{Ref} = (I_{PTAT} + I_{CTAT}) \cdot R_{Ref} \\
V_{Ref} = \frac{V_{PTAT}}{R_{PTAT}}\cdot R_{Ref} \\
V_{Ref} = \frac{V_{CTAT}}{R_{CTAT}}\cdot R_{Ref} \\
V_{Ref} = \frac{V_{PTAT} + V_{CTAT}}{R_{Ref}}\cdot R_{Ref} \\
V_{Ref} = V_{PTAT_1} - V_{PTAT_2} \\
V_{Ref} = V_{CTAT_1} - V_{CTAT_2}
\end{gather*}

\textbf{Formulas for $I_{PTAT}$, $I_{CTAT}$, $I_{Ref}$:}
\begin{gather*}
I_{PTAT} = \frac{V_{PTAT}}{R_{Ref}} \\
I_{PTAT} = \frac{I_{PTAT} \cdot R_{PTAT}}{R_{Ref}} \\
I_{CTAT} = \frac{V_{CTAT}}{R_{Ref}} \\
I_{CTAT} = \frac{I_{CTAT} \cdot R_{CTAT}}{R_{Ref}} \\
I_{CTAT} = \frac{V_{CTAT}}{R_{CTAT}} \\
I_{Ref} = \frac{V_{Ref}}{R_{Ref}}\\
I_{Ref} = \frac{V_{CTAT}}{R_{CTAT}}\\
I_{Ref} = \frac{V_{PTAT}}{R_{PTAT}}\\
I_{Ref} = \frac{V_{PTAT} + V_{CTAT}}{R_{Ref}}\\
I_{Ref}= \frac{I_{CTAT} \cdot R_{Ref}}{R_{CTAT}}\\
I_{Ref}= \frac{I_{PTAT} \cdot R_{Ref}}{R_{PTAT}}
\end{gather*}

\textbf{Formulas for $R_{PTAT}$, $R_{CTAT}$, $R_{Ref}$:}
\begin{gather*}
R_{PTAT} = \frac{V_{PTAT}}{I_{Ref}} \\
R_{PTAT} = \frac{V_{Ref} - V_{CTAT}}{I_{Ref}} \\
R_{PTAT} = \frac{V_{PTAT} \cdot R_{PTAT}}{V_{CTAT}}\\
R_{PTAT} = \frac{V_{PTAT} \cdot R_{CTAT}}{V_{CTAT}}\\
R_{CTAT} = \frac{V_{CTAT}}{I_{Ref}} \\
R_{CTAT} = \frac{V_{Ref} - V_{PTAT}}{I_{Ref}} \\
R_{Ref} = \frac{V_{Ref}}{I_{Ref}}\\
R_{CTAT} = \frac{V_{CTAT} \cdot R_{PTAT}}{V_{PTAT}}\\
R_{CTAT} = \frac{V_{CTAT} \cdot R_{CTAT}}{V_{CTAT}}\\
R_{Ref} = \frac{V_{PTAT}}{I_{PTAT}}
\end{gather*}

\section{Lecture 8 - Part of MidSem Project}
\begin{figure}[H]
    \centering
    \includegraphics[width=0.8\linewidth]{images/Lec_8_Q1_ckt.jpeg}
    \caption{Lecture 8 - Question 1: Circuit}
\end{figure}
\begin{figure}[H]
    \centering
    \includegraphics[width=1\linewidth]{images/Lec_8_Q1.jpeg}
    \caption{Lec 8: Question 1}
\end{figure}
\subsection{Slope of $V_{EB}$ vs Temperature curve of a BJT}
\subsection{Difference of $V_{EB}$ of two BJTs}
\subsection{Find the multiplication factor to equate PTAT and CTAT voltages}
\subsection{Simulate the BGR using ideal VCVS, VCVS for current mirror}
\section{Lecture 9 - MOSFET}
\subsection{NMOS I-V characteristics}
\begin{figure}[H]
    \centering
    \includegraphics[width=0.8\linewidth]{images/Lec_9_Q1.jpeg}
    \caption{Lecture 9 - Question 1}
\end{figure}
\subsection{PMOS I-V characteristics}
\begin{figure}[H]
    \centering
    \includegraphics[width=0.8\linewidth]{images/Lec_9_Q2.jpeg}
    \caption{Lecture 9 - Question 2}
\end{figure}

\section{Lecture 10 - Current Mirrors}
\subsection{Prove that Threshold voltage is linear and CTAT}
\begin{figure}[H]
    \centering
    \includegraphics[width=0.8\linewidth]{images/Lec_10_Q1.jpeg}
    \caption{Lecture 10 - Question 1}
\end{figure}

\subsection{Finding the current and sizign of the mosfet required to make $V_{GS}$ as Reference}
\begin{figure}[H]
    \centering
    \includegraphics[width=0.8\linewidth]{images/Lec_10_Q2.jpeg}
    \caption{Lecture 10 - Question 2}
\end{figure}

\subsection{PMOS Current Mirror - Simple, Widlar, Low swing, High swing}
\begin{figure}[H]
    \centering
    \includegraphics[width=0.8\linewidth]{images/Lec_10_Q3.jpeg}
    \caption{Lecture 10 - Question 3}
\end{figure}

\subsection{Plot $V_{EB}$ w.r.t Process}
\begin{figure}[H]
    \centering
    \includegraphics[width=0.8\linewidth]{images/Lec_10_Q4.jpeg}
    \caption{Lecture 10 - Question 4}
\end{figure}

\section{Lecture 11 - Startup Condition, Power Down Signal, Random Mismatch}
\begin{figure}[H]
    \centering
    \includegraphics[width=0.8\linewidth]{images/Lec_11_Questions.jpeg}
    \caption{Lecture 11 - Question 1 and Question 2}
\end{figure}
\subsection{Question 1}
\subsection{Question 2}

\section{Lecture 12 - Small Signal Analysis of Current Mirrors}
\subsection{Derivation of small signal equivalent}
\begin{figure}[H]
    \centering
    \includegraphics[width=0.8\linewidth]{images/Lec_12_Q1.jpeg}
    \caption{Lecture 12 - Question 1}
\end{figure}
\subsection{Finding the output resistance of the current mirror}
\begin{figure}[H]
    \centering
    \includegraphics[width=0.8\linewidth]{images/Lec_12_Q2.jpeg}
    \caption{Lecture 12 - Question 2}
\end{figure}

\section{Lecture 13 - Amplifiers}
\subsection{CS Amplifier with resistive load}
Design a CS amplifier with resistive load to get a gain of 20dB, $V_{DD} = 1.8V$, $I_Q$ = 1$\mu A$

\section{Lecture 14 - Single Stage Amplifiers}
\subsection{Drain feedback CS Amplifier}
Find conditions for $C_1$, $C_2$ to be short circuit in a drain feedback CS amplifier(NMOS based).
\subsection{Source feedback CS Amplifier}
Find conditions on $R_G$, $C_1$, $C_2$ in source feedback bias CS Amplifier(NMOS based).
\subsection{PMOS equivalent of CS Amplifiers}
Draw PMOS equivalent of voltage mode CS Amplifier (6), Current Mode CS Amplifier(3), CD and CG amplifier.

\section{Lecture 15 - Diffential Amplifiers}
\subsection{ICMR and OCMR of PMOS differential pair}
\begin{figure}[H]
    \centering
    \includegraphics[width=0.8\linewidth]{images/Lec_15_Q.png}
    \caption{Lecture 15 - Question 1}
\end{figure}

\section{Lecture 16 - Parasitic Capacitances and Poles}
\subsection{Bode Plot of NMOS CS Amp with capacitive load}
\begin{figure}[H]
    \centering
    \includegraphics[width=0.8\linewidth]{images/Lec_16_Q.png}
    \caption{Lecture 16 - Question 1}
\end{figure}
\section{Lecture 17}
\end{document}